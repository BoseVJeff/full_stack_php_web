\documentclass[12pt,a4paper]{report}

% For Image placement, esp. when there are multiple in a row
\usepackage{float}

\usepackage{fancyhdr}
\usepackage{lastpage}

\usepackage{tabularray}

% For table of Contents
\usepackage{multirow}
\usepackage{tabularx} % For flexible columns in ToC

% Used for page border on title page
% From https://tex.stackexchange.com/a/232684
\usepackage{tikz}
\usetikzlibrary{calc}
\usepackage{eso-pic}

% For page color for title page.
\usepackage{afterpage}
\usepackage[table]{xcolor}
\usepackage{pagecolor}

% To set padding in the title pages
\usepackage{setspace}
% To include images in the document
\usepackage{graphicx}
\graphicspath{{./assets/}}

% Set page margins
\usepackage[bindingoffset=0in]{geometry}
\geometry{margin=1in}

% \usepackage{imakeidx}
% \usepackage[normalem]{ulem}

% Format chapter, section, and other titles shown on the pages
% For centered titles, add `\centering` to the second argument
% Use `\theX` (X=chapter,section,etc) the get the chapter/section/etc number
\usepackage{titlesec}
\titleformat{\chapter}{\large\bfseries\centering}{}{1em}{\MakeUppercase}[\hrule]
\titleformat{\section}{\large\bfseries}{\thesection}{1em}{}
\titleformat{\subsection}{\bfseries}{\thesubsection}{1em}{}

% Setting ToC heading
\renewcommand*{\contentsname}{\uppercase{Index}}

\fancypagestyle{plain}{\fancyhead{}\renewcommand{\headrule}{}}

\usepackage{tocloft}
\renewcommand{\cftdot}{}
% \renewcommand{\cfttocdotsep}{\cftnodots}
\setcounter{tocdepth}{1}
% See pg 17 for `\ctfZ[pre/post]hook' in `tocloft` doc. It will prepend/append content/commands to the Z(toc,etc.) table
% Doc: https://in.mirrors.cicku.me/ctan/macros/latex/contrib/tocloft/tocloft.pdf
% e.g.: Use the following to set the toc into two column layout
% \RequirePackage{multicol}
% \renewcommand\cfttocprehook{\begin{multicols}{2}}
% \renewcommand\cfttocposthook{\end{multicols}}

% For custom layout, use `\newlistof` to generate a new custom ToC-like list that will funcion exactly same.
% See pg 13 of the `tocloft` for sample and example. Pay attention to the example and how it appends data to the entry. See the docs for `\theX` for a more involved example involving subitems.

% From: https://stackoverflow.com/a/877635
% Font family names: https://www.overleaf.com/learn/latex/Font_typefaces
\fontfamily{times}

\makeindex

% Use as \sqlref{table:students}{Students}
\newcommand{\sqlref}[2]{References \textbf{#2} table as defined in Table \ref{#1}}

\begin{document}
\newpagecolor{black}\afterpage{\restorepagecolor}
\thispagestyle{plain}
\begin{titlepage}
	\onehalfspacing
	\setlength{\baselineskip}{18pt}
	\color{white}
	\begin{center}
		\textbf{\LARGE{A}}

		\textbf{\LARGE{PROJECT REPORT}}

		\textbf{\LARGE{ON}}

		\Large{\underline{Cloud-based File Storage System}}
		\vspace{1cm}

		\textbf{\underline{SUBMITTED BY}}

		Vineet Maurya
		\vspace{1cm}

		\textbf{ACADEMIC YEAR 2024-25}


		\textbf{S.Y. B.C.A. SEM-3}
		\vspace{1cm}

		\textbf{\underline{UNDER THE GUIDANCE OF}}

		Ms Shraddha Doshi

		Ms Darshana Makani

		Ms Rupali Shinde

		\textbf{Navrachna University}
		\vspace{1cm}

		\textbf{\underline{SUBMITTED TO}}
		\vspace{0.5cm}

		\includegraphics{nuv_logo.png}
		\vspace{0.5cm}

		\textbf{Navrachna University}
	\end{center}
\end{titlepage}
\newpage
\thispagestyle{plain}
\onehalfspacing
\setlength{\baselineskip}{18pt}
\pagenumbering{gobble}
\color{black}
\begin{center}
	\textbf{\LARGE{A}}

	\textbf{\LARGE{PROJECT REPORT}}

	\textbf{\LARGE{ON}}

	\Large{\underline{Cloud-based File Storage System}}
	\vspace{1cm}

	\textbf{\underline{SUBMITTED BY}}

	Vineet Maurya
	\vspace{1cm}

	\textbf{ACADEMIC YEAR 2024-25}

	\textbf{S.Y. B.C.A. SEM-3}
	\vspace{1cm}

	\textbf{\underline{UNDER THE GUIDANCE OF}}

	Ms Shraddha Doshi

	Ms Darshana Makani

	Ms Rupali Shinde

	\textbf{Navrachna University}
	\vspace{1cm}

	\textbf{\underline{SUBMITTED TO}}
	\vspace{0.5cm}

	\includegraphics{nuv_logo.png}
	\vspace{0.5cm}

	\textbf{Navrachna University}
\end{center}
\newpage
\pagenumbering{roman}
\setcounter{page}{1}
\pagestyle{fancy}
\setlength{\headheight}{15pt}
\fancyhead[]{}
\fancyfoot[]{}
\fancyhead[L]{Enrollment No.: 23000068}
\fancyhead[C]{The Files Spot}
\fancyfoot[L]{Navarachna University}
\fancyfoot[C]{Page \thepage\ of \pageref{LastPage}}
\chapter*{Abstract}
This project represents the design and development of \textit{The Files Spot}, a file hosting and sharing service. It is aimed squarely at users who want more powerful functionality than what the existing file upload services provide. It leverages a flexible, web-based architecture to ensure that the service is available on whatever platform the user is on. The backend service is written in PHP and serves a basic HTML/CSS/JS frontend. The main features of this service include a very lightweight and responsive frontend for casual use and a flexible JSON API for those who want more control over their file management workflows.
\newpage
\chapter*{Acknowledgement}
I would like to take this opportunity to thank Mr Dhaval Mehta (Program Chair, BCA) for giving us the opportunity to work on a project of this scale.
I would also like to thank Ms Shraddha Doshi, Ms Rupali Shinde and Ms Darchana Makani. They were very helpful and paitent with us and were very helpful during the making of this project.
\newpage
\chapter*{Project Profile}
\begin{center}
	\begin{tabular}{|c|c|}
		\hline
		\multicolumn{2}{|c|}{\cellcolor{black} \color{white} \textbf{Student Information}} \\
		\hline
		\textbf{Name}            & \textbf{Enrollment Number}                              \\
		Vineet Maurya            & 23000068                                                \\
		\hline
		\multicolumn{2}{|c|}{\cellcolor{black} \color{white} \textbf{Project Details}}     \\
		\hline
		\textbf{Project Title}   & The File Spot - A Cloud based File Storage System       \\
		\hline
		\textbf{Duration}        & 3 months                                                \\
		\hline
		\textbf{Name of Project} & The Files Spot                                          \\
		\hline
		\textbf{Platform}        & Web                                                     \\
		\hline
		\textbf{Team Size}       & 1                                                       \\
		\hline
	\end{tabular}
\end{center}
\newpage
\pagenumbering{gobble}
\newpage
%\tableofcontents
\chapter*{Index}
% Page border (`*` variant to ensure that styling applies to this page only)
% From: https://tex.stackexchange.com/a/232684
\AddToShipoutPictureBG*{%
	\begin{tikzpicture}[overlay,remember picture]
		\draw[line width=2pt]
		($ (current page.north west) + (1cm,-1cm) $)
		rectangle
		($ (current page.south east) + (-1cm,1cm) $);
		\draw[line width=1pt]
		($ (current page.north west) + (1.1cm,-1.1cm) $)
		rectangle
		($ (current page.south east) + (-1.1cm,1.1cm) $);
	\end{tikzpicture}
}
\thispagestyle{plain}
\begin{tabularx}{\textwidth}{|c|X|c|}
	\hline
	\uppercase{Sr. No.} & \centerline{\uppercase{Title}}                                                & \uppercase{Page No.}                                \\
	\hline
	\multirow{4}{*}{1}  & \uppercase{SDLC Overview}                                                     & \pageref{cha:sdlc_overview}                         \\
	\hline
	                    & \quad \quad \quad 1.1 Requirement Gathering, Analysis, Definition, and Design & \pageref{sec:req}                                   \\
	\hline
	                    & \quad \quad \quad 1.2 Implementation, Testing                                 & \pageref{sec:impl}                                  \\
	\hline
	                    & \quad \quad \quad 1.3 Gantt Chart                                             & \pageref{sec:gantt}                                 \\
	\hline
	\multirow{7}{*}{2}  & \uppercase{Requirement Gathering and Analysis}                                & \pageref{sec:requirement_gathering_and_analysis}    \\
	\cline{2-3}
	                    & \quad \quad \quad 2.1 Organization Details                                    & \pageref{sec:organization_details}                  \\
	\cline{2-3}
	                    & \quad \quad \quad 2.2 Meetings                                                & \pageref{sec:meetings}                              \\
	\cline{2-3}
	                    & \quad \quad \quad 2.3 Data which will be Input into the System                & \pageref{sec:data_which_will_be_input_into_system}  \\
	\cline{2-3}
	                    & \quad \quad \quad 2.4 Data which will be Output from the System               & \pageref{sec:data_which_will_be_output_from_system} \\
	\cline{2-3}
	                    & \quad \quad \quad 2.5 Type of Project                                         & \pageref{sec:type_of_project}                       \\
	\cline{2-3}
	                    & \quad \quad \quad 2.6 Method of Collecting Data                               & \pageref{sec:method_of_collecting_data}             \\
	\hline
	\multirow{6}{*}{3}  & \uppercase{System Requirement Specifications}                                 & \pageref{cha:system_requirement_specifications}     \\
	\cline{2-3}
	                    & \quad \quad \quad 3.1 Introduction                                            & \pageref{sec:introduction}                          \\
	\cline{2-3}
	                    & \quad \quad \quad 3.2 Overall Description                                     & \pageref{sec:overall_description}                   \\
	\cline{2-3}
	                    & \quad \quad \quad 3.3 System Features                                         & \pageref{sec:system_features}                       \\
	\cline{2-3}
	                    & \quad \quad \quad 3.4 External Interface Requirements                         & \pageref{sec:external_interface_requirements}       \\
	\cline{2-3}
	                    & \quad \quad \quad 3.5 Other Non-Functional Requirements                       & \pageref{sec:other_non_functional_requirements}     \\
	\hline
	\multirow{6}{*}{4}  & \uppercase{System Analysis and Modelling}                                     & \pageref{cha:system_analysis_and_modelling}         \\
	\cline{2-3}
	                    & \quad \quad \quad 4.1 Use Case Diagram                                        & \pageref{sec:use_case_diagram}                      \\
	\cline{2-3}
	                    & \quad \quad \quad 4.2 ER Diagram                                              & \pageref{sec:normalisation_and_er_diagram}          \\
	\cline{2-3}
	                    & \quad \quad \quad 4.3 Data Dictionary                                         & \pageref{sec:data_dictionary}                       \\
	\cline{2-3}
	                    & \quad \quad \quad 4.4 Functional and Behaviour Modelling                      & \pageref{sec:functional_and_behavioural_modelling}  \\
	\cline{2-3}
	5                   & \uppercase{Test Cases}                                                        & \pageref{cha:test_cases}                            \\
	\hline
	6                   & \uppercase{Screenshots}                                                       & \pageref{cha:screenshots}                           \\
	\hline
	7                   & \uppercase{Limitations and Future Enhancements}                               & \pageref{cha:limitations_and_future_enhancements}   \\
	\hline
	8                   & \uppercase{Conclusion}                                                        & \pageref{cha:conclusion}                            \\
	\hline
	9                   & \uppercase{References and Bibilography}                                       & \pageref{cha:references_and_bibilography}           \\
	\hline
\end{tabularx}
% \normalsize
% \makeindex
\newpage
\pagenumbering{arabic}
\setcounter{page}{1}
\chapter{SDLC Overview}\label{cha:sdlc_overview}
\includegraphics{se-sdlc.png}
\section{Requirement Gathering, Analysis, Definition, and Design}\label{sec:req}
A major portion of the requirements were gathered via personal experiences and informal conversation with peers and other people.
Feedback left on other similar services online was also considered.
Collected requirements were then analysed and tested for feasablity.
Requirements deemed unviable (due to time or capabllity) were either rejected or moved to the "possible enhancements" section.
An example of a feature that was deemed unviable was S3 compatiblity.
While it did fit into the scope of the project, it was ultimately left out because of the sheer amount of work that would be required.

After the requirements were finalised, a project design was then considered.
The main goals of this project are simplicity and flexiblity.
The project design also places these concerns at its core.

The design was approached from two aspects - standard users (web UI, drag-and-drop) and powerusers (APIs, third-party clients).
For standard users, a simple and intuitive web UI is included.
For powerusers, we ensure that all equivalent functionality is also available via an API.

\section{Implementation, Testing}\label{sec:impl}
Implementation of this project started even before the design itself was finalised.
It was done in this manner so that the analysis and design phase could take advantage of the knowledge and experience gained during the actual implementation.

After the initial implementation was completed, test scaffolds were also set up.
At this initial stage, these consisted just unit tests.
As the implementation progressed, so did teh level of test coverage in this project.

\section{Project Timeline}\label{sec:gantt}
\begin{figure}[H]
	\includegraphics[width=\textwidth]{gantt.png}
\end{figure}

As can be seen from the above Gantt chart, a lot of project timeline was devoted to the planning and implementation.
There is also a lot of overlap for each task.
This was done in order to overcome the disadvantages of a traditional Waterfall model.

In a traditional Waterfall model, there is little-to-no feedback between each phase.
For a project as developer-centric as this, such rigidity is unadvisable.
Therefore, some overlap was planned so that mistakes made in the earlier phase of development can be corrected.

\newpage
\chapter{Requirement Gathering and Analysis}\label{sec:requirement_gathering_and_analysis}
\section{Organization Details}\label{sec:organization_details}

This service was made to fulfill part of the requirements for BCA III semester.

\begin{itemize}
	\item \textbf{Name of Organization}
	      Bachelor of Computer Applications, School of Engineering and Technology, Navarachna University, Vadodara

	\item \textbf{Brief Details of Organization}
	      The School of Engineering and Technology, Navrachna University is a place where students are taught details of various technical fields.
\end{itemize}

\section{Meetings}\label{sec:meetings}
Meetings with Shraddha Mam revealed valuable insights into the structre and contents of this document.

Meetings with Darshana Mam were crucial in finalising the functionality of the system made as a part of this project.

Rupali Mam's input helped shape the deployment story for this project.
\section{Data which will be Input into the System}\label{sec:data_which_will_be_input_into_system}
The service, by design, aims to require minimal data from the user. Aside from the username/password combo, only files themselves are the only other major form of data that most users are expected to input into the system.

Listed here, in no particular order, are all of the different kinds of data that can be input into the system:
\begin{itemize}
	\item \textbf{Account Details}

	      \textbf{Account Details} include data like usernames (arbritary string data), passwords (arbritary secure string data), and recovery email addresses (personal string data). The usernames are unique per-user and will be stored into the application database as-is. The passwords will be salted, hashed, and stored in a database and correlated with the username. The emails will be stored in a database as-is, correlated with the username. Note that emails are optional, and thus a username may/may not have an email address associated with it.

	\item \textbf{Files}

	      \textbf{Files} include the files themselves as well as any metadata that is associated with or generated from them. The metadata itself is associated with a file and is stored within the application database, and includes data like \texttt{upload\_date}, \texttt{uploader\_account}, etc. The files themselves are stored in a seperate storage space. This space is accessed only during file upload/download processes. All other interactions refer to the file's metadata stored within the application database.

	\item \textbf{Connection Metadata}

	      Some \textbf{Connection Metadata} about every file access is stored in the application database. This is mainly intended to control file access and to provide that data that powers the user's file access dashboard. For all users, this data includes access time. For logged in users, the data additionally includes the user's identity. For logged out users, this includes the connecting IP address. In both cases, simply visiting/accessing a page/file \textit{does not} set any kind of cookies or any other persistent storage on the visitor's user agent. Therefore, there is no data to see if the same user accessed the same page/file multiple times.
\end{itemize}

\section{Data which will be Output from the System}\label{sec:data_which_will_be_output_from_system}
Most of the data that will be output from the system will be the files that were previously uploaded by users. The system does not generate any user-facing files by itself.

The system does generate logs. These logs will be stored to a file on the server that is running the service. These logs will be accessible only from the server admin side.
\section{Type of Project}\label{sec:type_of_project}
This is a web-based service with seperate front and back ends. The back end is written in PHP with a MySQL database for data storage. The front end is a HTML/CSS/JS frontend that runs on the client's browser. Both sides communicate with each other over a RESTful JSON API.
\section{Method of collecting Data}\label{sec:method_of_collecting_data}
Data about other similar services was collected by using the services themselves and by reading about the experiences other people had when using the service. Particular attention was paid to the pain points that surfaced during use and from other users online.
\newpage
\chapter{System Requirement Specifications}\label{cha:system_requirement_specifications}
\section{Introduction}\label{sec:introduction}
The Files Spot (henceforth reffered to as TFS) is a simple file-storage service. It allows for simple and easy file backup, organization, and sharing.
\subsection{Purpose}
Many cloud storage solutions are currently available for general use for the public. These are usually either barebones storage services like Amazon S3 aimed at developers to build on top of, or customer-friendly services like Google Drive and Microsoft Onedrive. Unfortunately, services between these two extremes are few and far between. The Files Spot aims to fill this gap in the cloud storage space.

TFS is aimed squarely at power users who want a simple cloud storage service to store and backup their files in. Towards this end, TFS aims to deliver on two fronts - a simple user interface that makes it easy to get started, and a flexible API that allows for easy automation and integration with existing automation tools.
\subsection{Document Conventions}
The File Spot is henceforth reffered to as TFS in the rest of this document.

Users in this document is used to refer to anyone who uses the service, either via the provided web client or via the API.

In this document, clients refer to both the web client and to any service that consumes the service via the provied API. This includes, but is not limited to, various automation services that may use the service by scraping/parsing the web client instead of using the API.

All code in this document is \texttt{written like this}. It will be case-sensitive, and is intended to be parsed as-is.

Some portions of this document will require the user to replace the placeholder values with their own values.
The placeholders will be formatted as \texttt{\textsl{<placeholder>}}.
The user/reader is expected to replace the entire placeholder text (including the \texttt{<} and \texttt{>}) with the appropriate value of their own.

Some other portion of the document may require the user to put the result of some operation on some data and replace the placeholder with the result.
These operations are formatted \texttt{\textit{like this(}\textsl{<with the data indicated as placeholders>}\textit{)}}.
\subsection{Intended Audience and Reading Suggestions}
The intended audience for this documernt includes all stakeholders and any user who wishes to know more about the workings and design principles of the service. This should be especially useful for users who are wanting to know about the rationale behind certain decisions made during the development process.
\subsection{Project Scope}
The project explicitly aims to tackle the problem of uploading and downloading files. As such, support any and all file formats is within hthe scope of this project.

Security is provided via a basic login/token wall. The communication may be encrypted using standard HTTPS/TLS protocols, discussion about said protocols is out of the scope of this document.

File and data/metadata storage is a core part of this service and will be the main point of discussion in this document. Strategies to manage file access, data storage, and verification of file integrity will be an integral part of this project.

Encryption of files at rest is an explicit non-goal of this service. This is to avoid potential legal complexities regarding hosting of potentially illegal content. As such, all files uploaded to this service will be visible to the admins of the service. This is ordinarily meant to allow admins to comply with legal demands, but users should make sure that they don't upload any sensitive data to this service.

\section{Overall Description} \label{sec:overall_description}
On the user side, The Files Spot (TFS) is a simple web application that can be accessed from any web browser. It will have a UI for uploading new files to the cloud and a UI to manage files that have already been uploaded to the cloud. The UI will be very simple and minimalistic to ensure that it can be easily parsed by tools and other accessiblity tools.

On the backend, it will be a simple file storage service written in PHP that will manage files for multiple users. It will also allow users to share files - both to specific users or to the public at large. File management will be primarily accomplished via tags set by the user.

\subsection{Product Perspective}
\subsection{Product Features}
The web client is designed to be easy to use for all kinds of people on all kinds of connections. Specific care is taken to ensure that the design is easy to understand and interpret.

The underlying DOM of the web client is designed to be minimal and clean to ensure that various accessiblity services can easily surface relevant information to the user. Although this is a non-goal, an attempt will be made on a best-effort basis to ensure that the DOM itself remains relativey stable so that any scrapers that rely on it can remain functional as long as possible.

The API is meant to be very easy and intuitive to use. This is accomplished via the use of standard HTTP verbs (GET, POST, PUT, PATCH, DELETE) in the various file descriptors. The API will also be JSON-based and stateless (as far as possible) so that it is easy to inetgarte into existing CLI-based and app-based workflows.
\subsection{Use Cases and Characteristics}
The primary usecase for this service is to easily backup files to the cloud. In addition, the simple protocols make this a very accessible application.

Another use case for this service could be file transfer and sharing. It explicitly supports sharing with various people (or the world) to enable this usecase.
\subsection{Operating Environment}
This service can be run on any service that can run PHP.

People looking to deploy this service should first check what version of PHP their host supports. If a supported version is available, refer to the PHP deployment guides for that host.

Developers wanting to fork/make changes should refer to the comments in the source code. Additionally, we reccomend that they install just as a command runner to simplify their lives. This project makes heavy use of just recipies to automate mundate CLI tasks. For reference, this service was originally built and tested on PHP 8.3.10 running on Windows 10.
\subsection{Design and Implementation Constratints}
A major implementation constraint here is the need to make this service as user firendly as possible.

The web client aims to follow the principles of progressive enhancement. As a part of it, we ensure that the web app is useable in all three loading stages (HTML, HTML+CSS, HTML+CSS+Javascript). This therefore rules out JS-only frameworks like React, Vue, etc.

Another design decision to not use any large external libraries. This means that we mostly avoid large CSS libraries like Bootstrap. Instead, we opt for custom, minimal CSS and JS to power the site.

On the server side, the decision to avoid JS frameworks means that we need a language which can stringly augment the existing HTML instead of generating a new one everytime one is requested. This leaves us with two major, battle-tested options - Python (via Jinja2 templates in Django/Flask), or PHP. Here, we opt for PHP as it is a lighter, much simpler language to make a web server in. Additionally, PHP comes with a lot of server utilities built in (database methods, etc) which removes the need for many external dependencies.
\subsection{User Documentation}
\subsection{Assumption and Dependencies}
On the server side, the server requires the presence of a PHP runtime on the system. Additionally, the server requires access to a MySQL instance (either locally or on a seperate server) to serve as the application database.

The web client assumes that the user has a reasonably modern web browser and a HTTP(S) connection to the server that's running the backend service. It must me capable of making and parsing various kinds of HTTP requests (GET, POST, PUT, PATCH, DELETE) Additionally, the browser must be capable of parsing and displaying HTML/CSS, with JS required for additional functionality.

The API service simply requires a client that is capable of making various kinds of HTTP requests (GET,POST,PUT,PATCH,DELETE) and handling a JSON response.
\section{System Features}\label{sec:system_features}
% The system storage will be backed by a standard disk on the server. Each file will then also be recorded in a Postgresql database along with some metadata. This metadata will allow the server to verify that the file is in the state it was uploaded in.

\subsection{API Access}\label{ssec:api_access}

The primary feature of this service that differentiates it from other similar services is the presence of a simple, but powerful API.
This allows a user to integarte this service easily with their existing setups via something like a bash script that runs on a regular basis.

The API is exposed at the \texttt{/api/} endpoint.
Access to all but public files require the client to perform some form of authentication.

The easiest way is for the user to generate a user token from another client that the user is already authenticated on.
The token should then be passed with every request to identify the user.
The token can be included either as \texttt{Authorization: Bearer \textsl{<user-token>}} in the request header, or as \texttt{?access\_token=\textsl{<token>}} alongside other URL query parameters.

Another way is to use HTTP Basic Auth.
In this case, the username and password for the user are directly passed as a part of the request headers.
Note that this is heavily discouraged, as the credentials can then be revealed if the connection is insecure or if the user is a victim of a MitM attack.
To use this auth scheme, include \texttt{Authorization: Basic \textit{base64(}\textsl{<username>}:\textsl{<password>}\textit{)}} in the request headers.

Each endpoint supports some subset of the standard HTTP methods.
The information for each endpoint is defined as follows:

\begin{itemize}
	\item \texttt{/api/files/\textsl{<file>}}

	      \begin{itemize}
		      \item \texttt{GET}

		            Returns file contents with HTTP status \texttt{200 OK} if the file exists, and HTTP status \texttt{404 Not Found} if the file does not exist.
		            If the user is not authorised to access the file, an HTTP \texttt{401 Unauthorized} is returned instead.

		            By default, the response includes a \texttt{Content-Disposition: inline} header, indicating that the response is intended for display only.
		            Adding \texttt{?download=true} causes the reponse header to include \texttt{Content-Disposition: attatchment; filename="\textsl{<filename.ext>}"} instead, indicating that the file is intended to be downloaded to the client's system. \textsl{filename.ext} is the filename that the file was originally uploaded with.

		            By default, the entire file's contents are returned.
		            If the request contains the \texttt{Range: bytes=\textsl{<start-offset>}-\textsl{<end-offset>}} header, the contents of the file within the range \texttt{<start-offset>}-\texttt{<end-offset>} are returned with an HTTP \texttt{206} status code instead.
		            If the file does not exist, a HTTP \texttt{404} status code is returned instead.
		            \texttt{<start-offset>} is clamped to the file size, with a default value of 0.
		            \texttt{<end-offset>} is also clamped to the file size, with a default value of \texttt{file-size}.
		            If $\texttt{<start-offset>} > \texttt{<end-offset>}$, an error with HTTP status \texttt{400 Bad Request} is returned.

		      \item \texttt{PATCH}

		            Updates a part of the file.
		            The part of the file to be updated is indicated by the value of the \texttt{Content-Range} header (eg. \texttt{Content-Range: bytes }).

		            The request \textbf{must} contain the \texttt{Range: bytes=\textsl{<start-offset>}-\textsl{<end-offset>}} header.
		            The contents of the file within the range \texttt{<start-offset>}-\texttt{<end-offset>} are updated with the new contents.
		            If the file does not exist, a HTTP \texttt{404} status code is returned instead.
		            \texttt{<start-offset>} is clamped to the file size, with a default value of 0.
		            \texttt{<end-offset>} is also clamped to the file size, with a default value of \texttt{file-size}.
		            If \texttt{<start-offset>} \> \texttt{<end-offset>}, an error with HTTP status \texttt{400 Bad Request} is returned.

		      \item \texttt{DELETE}

		            Deletes the file.

		            If the file does not exist, returns a HTTP \texttt{404 Not Found} response.
		            If the user does not have the requisite permissions, \texttt{401 Unauthorized} is returned instead.

		            Note that a user cannot delete part of a file.
		            Therefore, any \texttt{Range:\textasteriskcentered} headers set on the request are ignored.
	      \end{itemize}

	\item \texttt{/api/user/config}

	      This endpoint returns any and all info about the user and any preferences that they may have set.
	      This includes their username, email (if set), and all app preferences that the system has about them.
	      Note that this endpoint \textbf{requires} an authenticated client, so any unauthenticated requests will simply return a \texttt{401 Unauthorized} response.

	      The base config for a user is of the following form:

	      %   \texttt{
	      %       \{
	      %       Name: \textsl{<name>},
	      %       Email: \textsl{<email>},
	      %       Subscription: \textsl{<subscription>}
	      %       \}
	      %   }

	      \texttt{\{ }

	      \texttt{\quad Name: \textsl{<name>}}

	      \texttt{\quad Email: \textsl{<email>}}

	      \texttt{\quad Subscription: \textsl{<subscription>}}

	      \texttt{\} }

	      Note that tokens are missing from this response.
	      This aims to make it more difficult for a malicious client to take over other tokens.

	      The remainder of this section assumes that requests are being made by an authenticated client.

	      \begin{itemize}
		      \item \texttt{GET}

		            Gets all user preferences.

		            If the user does not exist or if the token is invalid, returns \texttt{404 Not Found} instead.

		      \item \texttt{POST}

		            Send an new config for the user.

		            Any config that the user may have is replaced with the new one.

		      \item \texttt{PATCH}

		            Update the config with the new values.

		            Note that the rest of the config is left unchanged.

		      \item \texttt{DELETE}

		            Delete the user config.

		            Note that the system cannot ever delete a token without deleting the corresponding user.
		            Therefore, this action simply has the effect of resetting values back to default wherever possible.
	      \end{itemize}
\end{itemize}

\section{External Interface Requirements}\label{sec:external_interface_requirements}
The system requires unteraction with two major external systems: a database server, and a file server.

The database server is assumed to be wither MySQL, or another MySQL-compatible database server like MariaDB.
The main app server does not make any other assumptions about the database server.
Therefore, complications like database sharding and scaling are the responsiblity of the database server itself.

The service also assumes access to a file server.
By default, the files are stored on the same server as the app service itself i.e. the app server also acts as the default file server.
The service itself makes no assumptions about the file server itself.
Therefore, any file server complications (file duplication, seperate file storage service) will require the developer to write their own adapter for the server itself by \texttt{require}-ing and overriding the appropriate class.
The default file server included in this app stores the files in a seperate folder on the app server itself.
\subsection{User Interfaces}
The main user interface for this service is its default web UI.

The web UI is made with HTML, CSS, \& JS.
It aims to be simple, fast, and effective.
It allows a user to login, generate/manage user tokens for use with the API, and view/manage files that they have uploaded to the service through various means.
\subsection{Hardware Interfaces}
This app does not make use of any external hardware except those that power the server and the client.

All interaction with the hardware server happens through PHP.
Therefore, the hardware is assumed to be one that is capable of running a PHP server.

All interaction with the client's hardware happens through the user-agent that the user uses.
This hardware is mostly used to make requests, and store temporary data to ease the load on the app server.
\subsection{Software Interfaces}
The main software interface for this service is its API.

The API is a major part of this service's offering.
It is a simple, RESTful JSON API with support for standard HTTP verbs like \texttt{GET}, \texttt{POST}, \texttt{PATCH}, etc.

For a complete documentation, refer to section \ref{ssec:api_access} on page \pageref{ssec:api_access}.
\subsection{Communication Interfaces}
The app service is defined as an app-server model with a clear boundary between the server and the client.
Communication between the server and the client happens over a HTTP connection using a JSON API as described in section \ref{ssec:api_access} on page \pageref{ssec:api_access}.

\section{Other Non-Functional Requirements}\label{sec:other_non_functional_requirements}

\subsection{Performance Requirements}
The service can handle multiple users uploading and downlaoding files at the same time.
Although the exact number of simultaneous users depends on the exact configuration of the server hardware that powers this service, note that there is no restriction within the service framework request for limiting the number of concurrent users.
\subsection{Safety Requirements}
The app service makes no attempt to limit the kinds of content that a user can upload.
Therefore, it is the user's responsiblity to ensure that the content that they upload to or access from the service is safe.
\subsection{Security Requirements}
The service only guarantees access-level security i.e. it guarantees that an unintended user (except the server admin) will not be able to access any of he user's files and view/modify them.

The file data itself is also not encrypted both in transit (except using HSTS if configured) and at rest on the server (except when configured by the admin - see \ref{sec:external_interface_requirements} for a more detailled discussion.) by default.
Although the service makes an attempt to ensure that the files stored are returned as uploaded (and warns the user if a difference is found), note that is could be trivially bypassed by a dedicated attacker.
The suer must therefore employ other file integrity checks of their own (preferably out-of-band) to verify that the file downloaded has the intended contents.
\subsection{Software Quality Attributes}
Software quality is upheld by employing a series of tests, both automated and manual.
These tests aim to test every aspect of the system and ensure that it matches the expectatons defined in this document.
% \section{Other Requirements}
\newpage
\chapter{System Analysis and Modelling}\label{cha:system_analysis_and_modelling}
\section{Use case Diagram}\label{sec:use_case_diagram}
\begin{figure}[H]
	\centering
	\includegraphics[width=0.6\textwidth]{use-case.png}
\end{figure}
\section{E-R Diagram}\label{sec:normalisation_and_er_diagram}
\includegraphics[width=\textwidth]{erd.png}
\section{Data Dictionary}\label{sec:data_dictionary}
\begin{table}[h!]
	\centering
	\caption{User Table}
	\begin{tabular}{|c|c|c|}
		\hline
		\textbf{Field}    & \textbf{Datatype}        & \textbf{Constraints and Other Notes} \\
		\hline
		Id                & Integer                  & Primary key                          \\
		                  &                          & Autoincrement                        \\
		\hline
		Name              & String                   & Not null                             \\
		                  &                          & Unique                               \\
		\hline
		Password          & String                   & Non null                             \\
		                  &                          & Secure                               \\
		\hline
		Email             & String                   & Validated                            \\
		\hline
		Payment Id        & String                   & From payment provider                \\
		\hline
		Order Id          & String                   & From payemnt provider                \\
		\hline
		Payment Signature & String                   & From payment provider                \\
		\hline
		Payment Status    & String (Success|Failure) & Verified seperately                  \\
		\hline
		Payment Verfied   & Boolean                  & Authenticity of payment              \\
		\hline
	\end{tabular}
	\label{table:user}
\end{table}
\begin{table}[h!]
	\centering
	\caption{Token Table}
	\begin{tabular}{|c|c|c|}
		\hline
		\textbf{Field} & \textbf{Datatype} & \textbf{Constraints and Other Notes} \\
		\hline
		Id             & Integer           & Primary key                          \\
		               &                   & Autoincrement                        \\
		\hline
		Key            & String (UUID)     & Unique                               \\
		               &                   & Non null                             \\
		\hline
		Label          & String            & Human-readable                       \\
		               &                   & User defined                         \\
		\hline
		User Id        & Integer           & \sqlref{table:user}{User}            \\
		\hline
	\end{tabular}
	\label{table:token}
\end{table}
\begin{table}[h!]
	\centering
	\caption{File Table}
	\begin{tabular}{|c|c|c|}
		\hline
		\textbf{Field} & \textbf{Datatype} & \textbf{Constraints and Other Notes}                     \\
		\hline
		Id             & Integer           & Primary key, Autoincrement                               \\
		\hline
		Name           & String            & Unique, Non null, As stored on disk                      \\
		\hline
		User Id        & Integer           & \sqlref{table:user}{User}                                \\
		\hline
		Last Upload    & Timestamp         & Non-null                                                 \\
		\hline
		Original Name  & String            & As uploaded by user                                      \\
		\hline
		Size           & Integer           & Non null, Used to calculate storage consumption for user \\
		\hline
	\end{tabular}
	\label{table:file}
\end{table}
\begin{table}[h!]
	\centering
	\caption{Tag Table}
	\begin{tabular}{|c|c|c|}
		\hline
		\textbf{Field} & \textbf{Datatype} & \textbf{Constraints and Other Notes} \\
		\hline
		Id             & Integer           & Primary key, Autoincrement           \\
		\hline
		Label          & String            & Human-readable, User defined         \\
		\hline
		User Id        & Integer           & \sqlref{table:user}{User}            \\
		\hline
	\end{tabular}
	\label{table:tag}
\end{table}
\begin{table}[h!]
	\centering
	\caption{File Tag Table}
	\begin{tabular}{|c|c|c|}
		\hline
		\textbf{Field} & \textbf{Datatype} & \textbf{Constraints and Other Notes} \\
		\hline
		File Id        & Integer           & \sqlref{table:file}{File}            \\
		\hline
		Tag Id         & Integer           & \sqlref{table:tag}{Tag}              \\
		\hline
	\end{tabular}
	\label{table:file-tag}
\end{table}
\begin{table}[h!]
	\centering
	\caption{Subscription Table}
	\begin{tabular}{|c|c|c|}
		\hline
		\textbf{Field}  & \textbf{Datatype} & \textbf{Constraints and Other Notes}                \\
		\hline
		Id              & Integer           & Primary key, Autoincrement                          \\
		\hline
		Name            & String            & Human-readable                                      \\
		\hline
		Description     & String (Long)     & Description of subscription benefits                \\
		\hline
		Price           & Integer           & Positive non-zero integer                           \\
		                &                   & Defined in the smallest increment (eg. cents/paise) \\
		\hline
		Subscription Id & String            & Order item in payment processor                     \\
		\hline
		Currency        & String            & ISO code, Default: INR                              \\
		\hline
	\end{tabular}
	\label{table:subscription}
\end{table}
\section{Functional and Behavioural Modelling}\label{sec:functional_and_behavioural_modelling}
\subsection{Level 0 Context Diagram}
\includegraphics[width=\textwidth]{dfd-L0.png}
\subsection{Level 1 Data Flow Diagram}
\includegraphics[width=\textwidth]{dfd-L1-user.png}
\includegraphics[width=\textwidth]{dfd-L1-subscription.png}
\includegraphics[width=\textwidth]{dfd-L1-file.png}

\newpage
\chapter{Test Cases}\label{cha:test_cases}
% \begin{hyphenrules}{nohyphenation}
\begin{table}[H]
	\raggedright
	\centering
	\caption{User Login}
	\begin{tabular}{|p{0.33\textwidth}|p{0.33\textwidth}|p{0.33\textwidth}|}
		\hline
		Action                               & Expected Response                      & Actual Response                        \\
		\hline
		Username and Password correct        & Redirect user to dashboard             & Redirect user to dashboard             \\
		\hline
		Username incorrect, Password correct & Error - Incorrect username or password & Error - Incorrect username or password \\
		\hline
		Username correct, Password incorrect & Error - Incorrect username or password & Error - Incorrect username or password \\
		\hline
	\end{tabular}
\end{table}
% \begin{table}[H]
\begin{longtblr}[
	caption={File Access}
	]{
	colspec = {|X[c]|X[c]|X[c]|X[c]|},
	width = \textwidth,
	rowhead = 1,
	}
	% \raggedright
	% \centering
	% \caption{File Access}
	% \begin{tabular}{|p{0.25\textwidth}|p{0.25\textwidth}|p{0.25\textwidth}|p{0.25\textwidth}|}
	\hline
	API                                         & Action                                                                                                                                                                     & Expected Response                                                                                         & Actual Response                                                                                           \\
	\hline
	\SetCell[r=4]{c} \texttt{/api/files/<file>} & [GET] \texttt{<file>} does not exist                                                                                                                                       & HTTP 404 Not Found                                                                                        & HTTP 404                                                                                                  \\
	\hline
	                                            & [GET] \texttt{<file>} exists, user credentials invalid or missing                                                                                                          & HTTP 401 Unauthorized                                                                                     & HTTP 401 Unauthorized                                                                                     \\
	\hline
	                                            & [GET] \texttt{<file>} exists, user credentials valid, user not allowed to access file                                                                                      & HTTP 401 Unauthorized                                                                                     & HTTP 401 Unauthorized                                                                                     \\
	\hline
	                                            & [GET] \texttt{<file>} exists, user credentials valid, user allowed to access file, \texttt{Range} header not set                                                           & HTTP 200 OK, with the entire file in response body                                                        & HTTP 200 OK, with the entire file in response body                                                        \\
	\hline
	\SetCell[r=4]{c} \texttt{/api/files/<file>} & [GET] \texttt{<file>} exists, user credentials valid, user allowed to access file, \texttt{Range} header set to an invalid value                                           & HTTP 400 Bad Request                                                                                      & HTTP 400 Bad Request                                                                                      \\
	\hline
	                                            & [GET] \texttt{<file>} exists, user credentials valid, user allowed to access file, \texttt{Range} header set with $\texttt{start}<=0$, \texttt{end} unspecified or invalid & HTTP 206 Partial Content, with the entire file in response body                                           & HTTP 206 Partial Content, with the entire file in response body                                           \\
	\hline
	                                            & [GET] \texttt{<file>} exists, user credentials valid, user allowed to access file, \texttt{Range} header set with $\texttt{start}>0$, \texttt{end} unspecified or invalid  & HTTP 206 Partial Content, with file data starting from offset \texttt{<start>} to the end of the file     & HTTP 206 Partial Content, with file data starting from offset \texttt{<start>} to the end of the file     \\
	\hline
	                                            & [GET] \texttt{<file>} exists, user credentials valid, user allowed to access file, \texttt{Range} header set with $\texttt{start} > \texttt{end}$                          & HTTP 206 Partial Content, with file data starting from offset \texttt{<start>} to the end of the file     & HTTP 206 Partial Content, with file data starting from offset \texttt{<start>} to the end of the file     \\
	\hline
	\SetCell[r=2]{c} \texttt{/api/files/<file>} & [GET] \texttt{<file>} exists, user credentials valid, user allowed to access file, \texttt{Range} header set with $\texttt{start}>0, \texttt{end}>=\texttt{<file-size>}$   & HTTP 206 Partial Content, with file data starting from offset \texttt{<start>} to the end of the file     & HTTP 206 Partial Content, with file data starting from offset \texttt{<start>} to the end of the file     \\
	\hline
	                                            & [GET] \texttt{<file>} exists, user credentials valid, user allowed to access file, \texttt{Range} header set with $\texttt{start}>0, \texttt{end}<\texttt{<file-size>}$    & HTTP 206 Partial Content, with file data starting from offset \texttt{<start>} to the offset \texttt{end} & HTTP 206 Partial Content, with file data starting from offset \texttt{<start>} to the offset \texttt{end} \\
	\hline
	% \end{tabular}
\end{longtblr}
% \end{table}
% \end{hyphenrules}
\newpage
\chapter{Screenshots}\label{cha:screenshots}
\begin{figure}[H]
	\includegraphics[width=\textwidth]{login-empty.jpeg}
	\caption{Login Page}
\end{figure}
\newpage
\chapter{Limitations and Future Enhancements}\label{cha:limitations_and_future_enhancements}
One major limitation of this software is its dependency on the underlying system, making it more difficult to deploy on production servers.
Unlike some other modern systems that ship as a single binary, this software ships as a set of files that must be copied over to the server.
Additionally, said server also needs to be setup with the tools needed to get a typical PHP server running.

Another limitation of this software is its dependency on MySQL as its backend database.
This usually requires running MySQL as either a seperate process on the same server, or on a different server altogether.
This can result in increased server costs.

The server currently makes no attempt to identify potentially illegal content.
Currently, it is the responsiblity of the server admin to ensure that no illegal content is being hosted on this service.
But, in the future, system could be added to this service that filters out benign files and surfaces only the potentially problematic ones to the admin.
This would serve to reduce the admin's workload.

The central feature of this server is the level of access it provides via its API.
Unfortunately, this level of access also leaves it more vulnerable to bots.
For now, this service makes no attempt to identify if a user is a bot and instead, leaves it to the proxy server that will be running behind.

In the future, work could be done to ensure that this service's API is S3 compatible.
As a \textit{de facto} standard, a wide variety of applications support it.
Therefore, S3 compatiblity could make it even easier to support existing backup workflows.
\newpage
\chapter{Conclusion}\label{cha:conclusion}
The Files Spot is a web-based file hosting service that aims to be simple and easy to use.
It is aimed squarely at powerusers and other users who want more control over their file shares and backups.
\newpage
\chapter{References and Bibilography}\label{cha:references_and_bibilography}
\begin{enumerate}
	\item https://developer.mozilla.org/en-US/docs/Web/HTTP
	\item https://www.php.net/
	\item https://phptherightway.com/
	\item https://stackoverflow.com/
	\item https://tex.stackexchange.com/
	\item https://www.overleaf.com/
\end{enumerate}
\end{document}